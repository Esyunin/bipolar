\input{text/diss}
\begin{document}
\def\labauthors{Есюнин Д.В., Есюнин М.В.}
\def\labgroup{440}
\def\labnumber{2}
\def\labtheme{Измерение статических характеристик биполярного транзистора}
%\def\department{Кафедра электроники и квантовой радиофизики}
\input{text/titlepage}
\newpage
\section{Теоретическая часть}%
\subsection{Введение}%


Принцип действия биполярного транзистора состоит в управлении током неосновных носителей заряда, инжектируемых эмиттерным $p-n$ переходом
в базу и достигающих коллекторного $p-n$ перехода, включенного в запорном направлении. 

Управление током, протекающим через транзистор, достигается при помощи изменении высоты энергетических барьеров $p-n$ переходов:
прямосмещенного эмиттерного и обратносмещенного коллекторного. Биполярный транзистор является прибором, управляемым током -- малый ток
базы управляет большим током, протекающим из эмитера в коллектор.

\subsection{Устройство биполярного транзистора}%
\begin{figure}[h]
    \centering
    \includegraphics[width=\linewidth]{fig/1.jpg}
    \caption{}
    \label{fig:}
\end{figure}
Прибор представляет собой монокристалл, содержащий три полупроводниковых области с различным типом проводимости, которые
образуют между собой два $p-n$ перехода, а с наружными металлическими электродами -- омические контакты.

Как видно из рис. 1а ток, за исключением периферийных областей, течет перпендикулярно границам $p-n$ переходов. Обычно краевыми эффектами
на периферии структуры пренебрегают, так как толщина слоя базы много меньше её латеральных размеров. Идеализированная одномерная структура транзистора представлена на рис. 1.

Отметим две принципиальные конструктивно-технологические особенности транзисторов:
\begin{itemize}
    \item Малая толщина базы по сравнению с диффузионной длиной дырок $L_p$, являющихся в базе неосновными носителями.
    \item   Относительно малая степень легирования материла базы примесными атомами по сравнению с эмиттером и коллектора.
\end{itemize}


\subsubsection{Схема включение транзистора}%
Несмотря на то, что схема включения транзистора непосредственно не влияет на физику его работы, она определяет граничные условия на контактах.
На рис. 6 приведены две схемы включения транзистора: с общей базой (ОБ) и общим эмиттером (ОЭ).

\begin{figure}[h]
    \centering
    \includegraphics[width=\linewidth]{fig/2.jpg}
    \caption{}
    \label{fig:}
\end{figure}
\subsubsection{Зонная диаграмма транзистора в активном режиме}%
 Присоединим источники напряжения к клеммам транзистора. При
нормальном включении, обеспечивающим активный режим, на эмиттерный
переход должно быть подано прямое смещение, а на коллекторный переход
обратное. На рис. 2 показано включение источников по схеме с общей базой
при котором вывод базы является общим для обоих источников питания. При
малом уровне инжекции (т.е. вброса электронов и дырок соответственно
области $р$ и $n$-типа) электрическое поле вне перехода равно нулю. Тогда
на достаточном удалении от границ переходов носители находятся в состоянии
термодинамического равновесия, а уровни Ферми располагаются относительно
краев зон в соответствующих областях так же, как в равновесном транзисторе
(рис. 46). На рис. 76 изображена зонная диаграмма транзистора в активном
режиме работы.
Перепад уровней Ферми в областях $р-n$ переходов соответствие
приложенным к этим переходам напряжениям. Кроме того, приложенные
напряжения приводят к трансформации зонной диаграммы. 
\begin{itemize}
    \item Эмиттерный переход, находящийся под прямым смещением, сужается, а высота потенциального барьера в переходе уменьшается на $e_{0}U_{\text{эб}}$ ;
    \item Обратно-смещенный коллекторный переход расширяется, а высота потенциального барьера увеличивается на величину $e_0 U_{\text{кб}}$.
\end{itemize}

\begin{figure}[h]
    \centering
    \includegraphics[width=0.5\linewidth]{fig/3.jpg}
    \caption{}
    \label{fig:}
\end{figure}
%\subsubsection{Основные процессы в транзисторе, включенном по схеме ОБ}%

%Познакомимся с принципом действия транзистора на примере схемы с ОБ. Последняя, благодаря сравнительной простоте анализа, является основной при 
%рассмотрении физических процессов.

%Подадим в эмиттерную цепь входной сигнал $U_{\text{вх}}$, а в коллекторную цепь включим  \\



\label{sec:theory}
\section{Практическая часть}%
\subsection{Измерение входной характеристики транзистора $I_\text{б}=f(U_\text{б})$}
Для измерения входной характеристики транзистора собрали схему № 1,
изображенную на рис. (\ref{fig:8}). Сняли зависимость тока базового перехода от
напряжения на нем.
\begin{figure}[H]
	\centering
	\includegraphics[width=0.6\linewidth]{fig/shema1}
	\caption{Измерительная схема №1}
	\label{fig:8}
\end{figure}
% Table generated by Excel2LaTeX from sheet 'Лист1'
\begin{table}[htbp]
	\centering
%	\caption{Add caption}
	\begin{tabular}{|c|c|c|c|c|c|c|c|c|c|c|c|c|c|}
		\toprule
		$U_\text{б}, \text{ В}$  & 0.2   & 0.4   & 0.6   & 0.7   & 0.8   & 0.9   & 1     & 1.2   & 1.4   & 1.6   & 1.8   & 2     & 2.1 \\
		\midrule
		$I_\text{б}, \text{ мА}$ & 0     & 0     & 0.0024 & 0.017 & 0.062 & 190   & 274   & 1.75  & 2.7   & 3.6   & 4.6   & 5.54  & 6 \\
		\bottomrule
	\end{tabular}%
%	\label{tab:addlabel}%
\end{table}%
\begin{figure}[H]
	\centering
	\includegraphics[width=\linewidth]{plot/plot1}
	\caption{входная характеристика транзистора $I_\text{б}=f(U_\text{б})$}
	\label{fig:9}
\end{figure}
\subsection{Измерение переходных характеристик транзистора $I_\text{к}=f(U_\text{б})$}
\begin{figure}[H]
	\centering
	\includegraphics[width=0.6\linewidth]{fig/shema2}
	\caption{Измерительная схема №2}
	\label{fig:10}
\end{figure}
Для выполнения этого задания собрали измерительную схему
№ 2 (\ref{fig:10}). Провели измерение семейства переходных характеристик при
напряжениях на коллекторе транзистора 0,5 В, 1 В и 5 В.
% Table generated by Excel2LaTeX from sheet 'Лист2'
\begin{table}[htbp]
	\centering
%	\caption{Add caption}
	\begin{tabular}{|c|c|c|c|c|c|}
		\toprule
		\multicolumn{2}{|c|}{$U_k=0.5\text{ В}$} & \multicolumn{2}{c|}{$U_k=1\text{ В}$} & \multicolumn{2}{c|}{$U_k=1.5\text{ В}$} \\
		\midrule
		$U_\text{б}, \text{ В}$  & $I_k, \text{ мА}$ & $U_\text{б}, \text{ В}$  & $I_k, \text{ мА}$ & $U_\text{б}, \text{ В}$  & $I_k, \text{ мА}$ \\
		\midrule
		1     & 0.02  & 1     & 0.02  & 1     & 0.02 \\
		\midrule
		1.5   & 0.1   & 1.5   & 0.13  & 1.5   & 0.12 \\
		\midrule
		2     & 0.25  & 2     & 0.26  & 2     & 0.25 \\
		\midrule
		2.5   & 0.38  & 2.5   & 0.43  & 2.5   & 0.41 \\
		\midrule
		3     & 0.59  & 3     & 0.62  & 3     & 0.62 \\
		\midrule
		3.5   & 0.8   & 3.5   & 0.82  & 3.5   & 0.82 \\
		\midrule
		4     & 1     & 4     & 1.04  & 4     & 1.04 \\
		\midrule
		4.5   & 1.23  & 4.5   & 1.26  & 4.5   & 1.27 \\
		\midrule
		5     & 1.45  & 5     & 1.52  & 5     & 1.51 \\
		\midrule
		6     & 1.82  & 6     & 2     & 6     & 2.06 \\
		\midrule
		7     & 2.08  & 7     & 2.56  & 7     & 2.6 \\
		\midrule
		8     & 2.22  & 8     & 3.06  & 8     & 3.14 \\
		\midrule
		9     & 2.34  & 9     & 3.64  & 9     & 3.69 \\
		\midrule
		10    & 2.41  & 10    & 4.18  & 10    & 4.24 \\
		\bottomrule
	\end{tabular}%
%	\label{tab:addlabel}%
\end{table}%
\begin{figure}[H]
	\centering
	\includegraphics[width=\linewidth]{plot/plot2}
	\caption{переходная характеристика транзистора $I_\text{к}=f(U_\text{б})$}
	\label{fig:11}
\end{figure}
\subsection{Измерение выходных характеристик транзистора $I_\text{к}=f(U_\text{к})$}
Выполнение этого задания производится при включении транзистора по
схеме №2 (\ref{fig:10}). Были проведены измерения семейства выходных характеристик
при токах базы транзистора 20 мкА, 40 мкА, 60 мкА и 80 мкА.
% Table generated by Excel2LaTeX from sheet 'Лист3'
\begin{table}[htbp]
	\centering
%	\caption{Add caption}
	\begin{tabular}{|c|c|c|c|c|c|c|c|}
		\toprule
		\multicolumn{2}{|c|}{$I_\text{б}=20\text{ мкА}$} & \multicolumn{2}{c|}{$I_\text{б}=40\text{ мкА}$} & \multicolumn{2}{c|}{$I_\text{б}=60\text{ мкА}$} & \multicolumn{2}{c|}{$I_\text{б}=80\text{ мкА}$} \\
		\midrule
		$U_\text{к}, \text{ В}$  & $I_k, \text{ мА}$ & $U_\text{к}, \text{ В}$  & $I_k, \text{ мА}$ & $U_\text{к}, \text{ В}$  & $I_k, \text{ мА}$ & $U_\text{к}, \text{ В}$  & $I_k, \text{ мА}$ \\
		\midrule
		0.2   & 0.09  & 0.2   & 0.13  & 0.2   & 0.26  & 0.2   & 0.37 \\
		\midrule
		0.3   & 0.16  & 0.3   & 0.48  & 0.3   & 0.64  & 0.3   & 0.8 \\
		\midrule
		0.4   & 0.24  & 0.4   & 0.84  & 0.4   & 1.22  & 0.4   & 1.6 \\
		\midrule
		1     & 0.25  & 0.5   & 1     & 0.5   & 1.81  & 0.5   & 2.01 \\
		\midrule
		2     & 0.26  & 1     & 1.03  & 0.6   & 1.96  & 0.6   & 2.87 \\
		\midrule
		3     & 0.27  & 2     & 1.07  & 1     & 2.03  & 1     & 3.07 \\
		\midrule
		4     & 0.27  & 3     & 1.1   & 2     & 2.1   & 2     & 3.18 \\
		\midrule
		5     & 0.28  & 4     & 1.12  & 3     & 2.17  & 3     & 3.28 \\
		\midrule
		6     & 0.29  & 5     & 1.16  & 4     & 2.22  & 4     & 3.36 \\
		\midrule
		7     & 0.3   & 6     & 1.2   & 5     & 2.28  & 5     & 3.45 \\
		\midrule
		8     & 0.32  & 7     & 1.25  & 6     & 2.35  & 6     & 3.56 \\
		\midrule
		9     & 0.35  & 8     & 1.36  & 7     & 2.47  & 7     & 3.72 \\
		\midrule
		10    & 0.43  & 9     & 1.55  & 8     & 2.67  & 8     & 4 \\
		\midrule
		&       & 10    & 2.02  & 9     & 3.07  & 9     & 4.56 \\
		\midrule
		&       &       &       & 10    & 3.94  & 9.2   & 4.76 \\
		\midrule
		&       &       &       &       &       & 9.4   & 4.92 \\
		\midrule
		&       &       &       &       &       & 9.6   & 5.1 \\
		\midrule
		&       &       &       &       &       & 9.8   & 5.36 \\
		\midrule
		&       &       &       &       &       & 10    & 5.66 \\
		\bottomrule
	\end{tabular}%
%	\label{tab:addlabel}%
\end{table}%
\begin{figure}[H]
	\centering
	\includegraphics[width=\linewidth]{plot/plot3}
	\caption{выходная характеристика транзистора $I_\text{к}=f(U_\text{к})$}
	\label{fig:12}
\end{figure}
%\label{sec:practice}
\subsection{Расчёт параметров транзистора}%
%\begin{figure}[H]
%    \begin{minipage}{0.49\linewidth}
%        \centering
%        \includegraphics[scale=1]{fig/1.pdf}
%        \newline
%        (a)
%    \end{minipage}
%    \label{fig:1}
%    \hfill
%    \begin{minipage}{0.49\linewidth}
%        \centering
%        \includegraphics[scale=1]{fig/2.pdf}
%        \newline
%        (b)
%    \end{minipage}
%    \label{fig:1}
%    \centering
%   \begin{minipage}{0.49\linewidth}
%        \centering
%        \includegraphics[scale=1]{fig/3.pdf}
%        \newline
%        (c)
%    \end{minipage}
%    \caption{Статические характеристики биполярного транзистора:
%    (a) входная характеристика, (b) переходная характеристика, (c) выходная характеристика}
%\end{figure}


Рассмотренные в теоретической части эквивалентные схемы не являются единственно возможными. В литературе можно встретить множество других схем, в частности, П-образные схемы.
С точки зрения схемотехники выбор конкретной схемы не имеет существенного значения. Достаточно представить транзистор в виде некоторого бесструктурного 
четырехполюсника, и задать связи между входными и выходными величинами.

В приближении малого сигнала 4-х полюсник является линейным и упомянутым связям соответствует система двух линейных уравнений.
Естественно, что коэффициенты уравнений (параметры 4-х полюсника) зависят не только от физических свойств транзистора и режима,
но также от его схемы включения и выбора каких-то двух величин из 4-х в качестве управляющих переменных.

Преимуществом такого подхода является устранение произвола, связанного с выбором той или иной эквивалентной схемы, т.к. величины
параметров определяются непосредственно из уравнений транзистора.

Рассмотрим для иллюстрации определение $h$-параметров транзистора для включения с общим эмиттером. Для
$h$-системы в качестве независимых (управляющих) переменных выбираются входной ток и выходное напряжение.
В результате уравнения линейного 4-х полюсника имеет вид:

\begin{equation}
    \begin{aligned}
        \label{eq:50}
        U_{\text{вх}} = h_{11} i_{\text{вх}} + h_{12} U_{\text{вх}},\\
        i_{\text{вых}} = h_{21} i_{\text{вх}} + h_{22} U_{\text{вых}}.
    \end{aligned}
\end{equation}
Из \eqref{eq:50} следует, что
\begin{equation}
    \begin{aligned}
        \label{eq:h}
        h_{11}= \qty( \frac{U_{\text{вх}}}{U_{\text{вых}}})\eval_{i_{\text{вх}}=0} \quad
        h_{12}= \qty( \frac{U_{\text{вх}}}{i_{\text{вх}}})\eval_{U_{\text{вых}}=0} \\
        h_{21}= \qty( \frac{i_{\text{вых}}}{i_{\text{вх}}})\eval_{U_{\text{вых}}=0} \quad
        h_{22}= \qty( \frac{i_{\text{вых}}}{U_{\text{вых}}})\eval_{i_{\text{вх}}=0}
    \end{aligned}
\end{equation}

Согласно \eqref{eq:h}, $h_{11}$ имеет смысл входного сопротивления 4-х полюсника при закороченном выходе; $h_{12}$-- коэффициент обратной связи по напряжению,
при разомкнутом входе; $h_{21}$-- коэффициент усиления по току при закороченном выходе; $h_{22}$-- выходная проводимость при разомкнутом входе.
Для системы с ОЭ:
\begin{equation}
    \label{eq:52}
    \begin{aligned}
        i_{\text{вх}}=i_{\text{б}} \quad
        i_{\text{вых}}=i_{\text{к}}\\
        U_{\text{вых}}=U_{\text{к}}+U_{\text{э}}=U_{\text{к}}^*
    \end{aligned}
\end{equation}
Отсюда можно получить $h$-параметры через основные параметры транзистора:
\begin{equation}
    \begin{aligned}
        \label{eq:53}
        &h_{11}=r_{\text{б}} + \frac{r_{\text{э}}}{1- \alpha},
        &h_{12}=\frac{r_{\text{э}}}{2(1-\alpha)r_{\text{к}}},\\
        &h_{21}=\beta=\frac{\alpha}{1-\alpha},
        &h_{22}= \frac{1}{(1-\alpha)r_{\text{к}}}
    \end{aligned}
\end{equation}

Формулы \eqref{eq:52} служат в качестве исходных при измерениях $h$-параметров. Согласно \eqref{eq:52} они определяются
из опытов при условии короткого замыкания на выходе или холостого на входе. Поскольку выходная цепь в схеме с ОЭ
является высокоомной, а входная, наоборот -- низкоомной, указанные эксперименты не вызывают затруднений. Именно
поэтому $h$-система наиболее удобна для схем с ОЭ и ОБ. Отметим, что величины $h$- одного и того же транзистора 
при различных схемах включения транзистора также различны.

$h$-параметры будем рассчитывать при $I_\text{б} = 40 \text{ мкА}, U_\text{к} = 5 \text{В}$. Из прямой характеристики перехода база-эмиттер мы можем найти сопротивление базы $R_\text{б} = h_{11} = \frac{\Delta U_\text{б}}{\Delta I_\text{б}}$
\begin{equation}
\begin{aligned}
\label{eq:h}
& h_{11}= \qty( \frac{U_{\text{б}}}{i_{\text{б}}})\eval_{U_{\text{к}}=5 \text{ В}}= \frac{(2 -1.2) \text{ В}}{(5.54-1.75)\text{ мА}}\approx 211 \text{ Ом}\\
& h_{12}= \qty( \frac{U_{\text{б}}}{U_{\text{к}}})\eval_{i_{\text{б}}=40\text{ мкА}}=\frac{i_{\text{б}}\cdot R_{\text{б}}}{U_{\text{к}}}=\frac{40 \text{ мкА}\cdot 211 \text{ Ом}}{5 \text{ В}}\approx0,042 \\
& h_{21}= \qty( \frac{i_{\text{к}}}{i_{\text{б}}})\eval_{U_{\text{к}}=5 \text{ В}}=\frac{1.16 \text{ мА}}{40 \text{ мкА}}= 29 \\
& h_{22}= \qty( \frac{i_{\text{к}}}{U_{\text{к}}})\eval_{i_{\text{б}}=40\text{ мкА}}=\frac{1.16 \text{ мА}}{5 \text{ В}}= 0.23 \text{ Ом}^{-1}
\end{aligned}
\end{equation}
По результатам проведённых измерений были рассчитаны и построены зависимости коэффициента передачи тока от напряжения коллектора при токах 
коллектора 2 мА и 5 мА. 
Для транзистора, включенного по схеме с общим эмиттером, уравнение \eqref{eq:h} примет вид
\begin{equation}
    \begin{aligned}
        \label{eq:}
       & h_{11}= \qty( \frac{U_{\text{б}}}{U_{\text{э}}})\eval_{i_{\text{к}}=0}  
       & h_{12}= \qty( \frac{U_{\text{к}}}{i_{\text{к}}})\eval_{U_{\text{э}}=0} \\
       & h_{21}= \qty( \frac{i_{\text{э}}}{i_{\text{к}}})\eval_{U_{\text{э}}=0}  \quad
       & h_{22}= \qty( \frac{i_{\text{э}}}{U_{\text{э}}})\eval_{i_{\text{к}}=0}
    \end{aligned}
\end{equation}

Коэффициент передачи по току для схемы с ОЭ
\begin{equation}
    \label{eq:}
    K_{i}(U_{\text{к}}) = \frac{i_{\text{к}}}{i_{\text{б}}} =\frac{\alpha}{\alpha-1} , \text{ где } \alpha=\frac{i_{\text{к}}}{i_{\text{б}}+i_{\text{к}}}
\end{equation}
По результатам измерений, приведенных в предыдущем параграфе, были рассчитаны коэффициенты передачи тока при токе коллектора 2 мА и 4 мА.
\begin{equation}
\begin{aligned}
\label{eq:}
\alpha_1 =\frac{2 \text{ мА}}{60 \text{ мкА}+2\text{ мА}}=0.971, K_1=332\\
\alpha_2 =\frac{4 \text{ мА}}{80 \text{ мкА}+4\text{ мА}}=0.980, K_2=499\\
\end{aligned}
\end{equation}

\subsection{Измерение коэффициента усиления однокаскадного усилителя}%
В данном опыте была собрана схема однокаскадного усилителя (см. рис. \ref{fig:7}) и произведено измерение его коэффициента 
усиления в зависимости от частоты входного сигнала.
Амплитуда сигнала с генератора соответственно равна $A=16,1\text{ мВ}$, напряжение $Е_1 = 6 \text{ В}$, $Е_2 = 8.1 \text{ В}$, сила тока коллектора $I_\text{ к} = 3,07 \text{ мА}$
\begin{figure}[H]
    \centering
    \includegraphics[width=0.6\linewidth]{fig/shema3}
    \caption{Однокаскадный усилитель}
    \label{fig:7}
\end{figure}

\begin{figure}[H]
    \centering
    \includegraphics[scale=1]{plot/plot4}
    \caption{Коэффициент усиления однокаскадного усилителя. Пунктиром отмечена высота, соответствующая уровню $\frac{1}{2}$ относительно максимума}
    \label{fig:5}
\end{figure}

Полосой пропускания будем считать ординаты $K(\nu)$, соответствующие уровню 0.5 от максимума функции  $K(\nu)$.
Получаем полосу:
\begin{equation}
    \label{eq:}
         \nu_{\text{мин}} = 120 \text{ Гц}, \quad \nu_{\text{макс}} = 500 \text{ кГц} \\
\end{equation}
\begin{equation}
    \label{eq:}
     \nu_{\text{min}} < \nu < \nu_{\text{макс}} 
\end{equation}

\subsection{Измерение времени переключения транзистора}%
Используйте измерительную схему № 3(\ref{fig:7}). Установили для транзистора
режим отсечки: напряжение $E_1 = 6 \text{ В}$, ток базы – ноль, при этом ток коллектора
должен быть равен нулю, напряжение на коллекторе около 6 В. Подали с
генератора прямоугольный сигнал «меандр» частотой 120...150 кГц,
напряжением 2...3 В. Получите на осциллографе выходной сигнал.
Подстройте уровень входного сигнала так, чтобы транзистор
переключался из режима отсечки в режим насыщения $A=8,1\text{ мВ}$. Измерили зависимость
времени переключения транзистора из режима отсечки в режим насыщения и
из режима насыщения в режим отсечки от тока базы транзистора.
% Table generated by Excel2LaTeX from sheet 'Лист5'
\begin{table}[htbp]
	\centering
%	\caption{Add caption}
	\begin{tabular}{|c|c|c|}
		\toprule
		$I_\text{б}, \text{ мкА}$ & $\tau_{\text{о}\to\text{н}}, \text{ мкс}$ & $\tau_{\text{н}\to\text{о}}, \text{ мкс} $\\
		\midrule
		0     & 2.5   & 1.5 \\
		\midrule
		6     & 2     & 1.5 \\
		\midrule
		22    & 1.5   & 1 \\
		\midrule
		52    & 1     & 1 \\
		\midrule
		100   & 0.7   & 1 \\
		\bottomrule
	\end{tabular}%
%	\label{tab:addlabel}%
\end{table}%

\begin{figure}[H]
    \centering
    \includegraphics[scale=1]{plot/plot5}
    \caption{Зависимость времени переключения транзистора от тока базы транзистора}
    \label{fig:6}
\end{figure}
\section{Вывод}
Были изучены некоторые элементы теории $p-n-p$ переходов, сняты экспериментальные данные, по которым построены входная, переходная и выходная характеристики
транзистора. По результатам измерений ВАХ найдены: коэффициент передачи тока и времена переключения транзистора из режима насыщения в режим отсечки.
\end{document}
